\chapter{Método de uso}\label{chapter:Tutorial}

Este capítulo explica los pasos que debe seguir un usuario para resolver variantes de VRP utilizando el sistema que se implementó en este trabajo. Esto se hará a través de un ejemplo de resolución de un problema CVRP arbitrario. Para esto se define el problema en \ref{4-problem}, se define la solución y se construye el grafo de evaluación en \ref{4-solution} y se generan las funciones de exploración en \ref{4-generator}. En \ref{4-union} se presenta el código completo.

\section{Definición del problema}\label{4-problem}

El primer paso para resolver un VRP usando este sistema, es definirlo usando las clases que lo representan a él y a sus características. Esto se mostrará a través de un ejemplo de resolución de un CVRP con seis clientes. En este caso, se debe crear un objeto de tipo \textbf{basic-cvrp-client} (cliente con demanda) por cada cliente, uno de tipo \textbf{basic-depot} (depósito) y una matriz de distancias de tamaño $6x6$. 

El siguiente código se usa para definir el problema.

 \begin{lstlisting}
(defparameter c1 (basic-cvrp-client 1 1))
(defparameter c2 (basic-cvrp-client 2 1))
(defparameter c3 (basic-cvrp-client 3 4))
(defparameter c4 (basic-cvrp-client 4 3))
(defparameter c5 (basic-cvrp-client 5 2))
(defparameter c6 (basic-cvrp-client 6 1))

(defparameter d0 (basic-depot))

(defparameter dist-mat #2A((0 1 2 3 4 5 6)
										(1 0 5 2 1 3 2)
										(2 5 0 2 2 2 2)
										(3 2 2 0 1 2 1)
										(4 1 2 1 0 2 3)
										(5 3 2 2 2 0 1)
										(6 2 2 1 3 1 0)))

(defparameter problem (basic-cvrp-problem 
										:id 1 
										:clients (list c1 c2 c3 c4 c5 c6)
										:depot d0 
										:distance-matrix dist-mat 
										:capacity 20))
\end{lstlisting}

En las líneas 1 a 6 se definen los clientes, estos reciben dos parámetros, \textit{id} y \textit{demanda} respectivamente. Los vehículos se definen en las líneas 8 y 9, reciben también dos parámetros \textit{id} y \textit{capacidad} respectivamente. Luego se declara el depósito (línea 11), la matriz de distancias (línea 13) y se crea la instancia del problema (línea 21).

Una vez se definió el problema, se debe crear una solución inicial como se muestra en la siguiente sección.

\section{Solución inicial y Grafo de Evaluación}\label{4-solution}

La solución inicial se utiliza para construir e inicializar el grafo. En el caso de CVRP una solución está formada por una lista de rutas que son instancias de la clase \textbf{route-for-simulation}. El siguiente código muestra cómo se crea una solución:


\begin{lstlisting}
(defparameter r1 (route-for-simulation 
							:id 1
							:vehicle (cvrp-vehicle 1 20) 
							:depot d0
							:clients (list c1 c2 c3 (clone d0))
							:previous-client (clone d0)))
(defparameter r2 (route-for-simulation 
							:id 2 
							:vehicle (cvrp-vehicle 2 20)
							:depot d0
							:clients (list c4 c5 c6 (clone d0))
							:previous-client (clone d0)))

(defparameter s1 (basic-solution
							:id 1 
							:routes (list r1 r2)))
\end{lstlisting}

Las rutas se definen como instancias de la clase \textbf{rute-for-simulation} (líneas 1 y 7). Esta clase hereda de la clase \textbf{basic-route} y la extiende con la propiedad \textit{previous-client}. Esta propiedad se inicializa con una referencia al depósito a partir de la cual se crea el primer nodo de la ruta en el grafo de evaluación y simplifica la implementación del código de evaluación de la solución. Nótese que las rutas del grafo de evaluación comienzan y terminan en el depósito que se debe clonar para tener, en cada posición, nodos distintos.

Luego se debe inicializar el grafo con la función \textbf{init-graph}.

\begin{lstlisting}
(defparameter graph (init-graph s1))
\end{lstlisting}

Una vez que el grafo está inicializado, se le deben agregar las operaciones y nodos que mantienen valores. Esto se consigue mediante el código de evaluación de la solución inicial.

\begin{lstlisting}
(def-var total-distance 0 graph)
(loop for r in (routes s1) do
	(def-var route-distance 0 graph)
	(def-var route-demand (capacity (vehicle r)) graph) 
	(loop for c in (clients r) do 
		(increment-distance (previous-client r) c route-distance dist-mat graph)
		(decrement-demand c route-demand graph) 
		(setf (previous-client r) c)
	(increment-value total-distance route-distance graph)
	(apply-penalty route-demand total-distance 10 graph)
(return-value total-distance graph))
\end{lstlisting}

En la línea 1 se usa \textbf{def-var} para inicializar la variable que almacena el costo total de la solución. Luego se analizan las rutas de la solución. Por cada ruta se define una variable nueva que almacena su costo (línea 3) y otra que almacena la capacidad restante de su vehículo (línea 4). Entonces se analizan los clientes de la ruta actual. Se incrementa el costo de la ruta en una cantidad igual a la distancia del cliente actual y el cliente previo (el cliente previo inicial de la ruta es el depósito), y se disminuye la capacidad del vehículo en una cantidad igual a la demanda del cliente (líneas 6 y 7 respectivamente). Después de analizar cada ruta se aumenta el costo total de la solución en una cantidad igual a los costos de las mismas y se aplica penalización en caso de que un vehículo haya alcanzado una capacidad restante negativa (líneas 9 y 10). Finalmente se retorna la distancia total.

La siguiente sección ejemplifica cómo se generan las funciones de exploración a partir de criterios de vecindad y estrategias de exploración y selección.

\section{Generación de funciones de exploración}\label{4-generator}

Una vez que se tiene el grafo de evaluación de una solución inicial, para resolver el problema con una búsqueda local se debe definir cómo explorar la vecindad.

Las estrategias de exploración y selección se definen como clases cuyas instancias se pasan como parámetros a la función generadora. Se utilizan métodos que reciben estas instancias y generan código a partir de especializaciones de los tipos de los argumentos recibidos.

A continuación se presentan algunas clases que representan estrategias de exploración y selección:\\

\textit{Exploración:}
\begin{itemize}
	\item \textbf{exhaustive-neighborhood-search-strategy}: Exploración exhaustiva.
	\item \textbf{random-neighborhood-search-strategy}: Exploración aleatoria.
\end{itemize}

\textit{Selección:}
\begin{itemize}
	\item \textbf{best-improvement-search-strategy}: Selección de mejor solución.
	\item \textbf{first-improvement-search-strategy}: Selección de primera mejora.
	\item \textbf{random-improvement-with-candidates-selection-strategy}: Selección aleatoria de una mejora entre los elementos de una lista con las mejoras encontradas.
	\item \textbf{random-improvement-selection-strategy}: Selección aleatoria de una mejora en base a una probabilidad.
\end{itemize}

Durante la definición de cada clase se crea un parámetro global que almacena una instancia de su respectiva clase. Por ejemplo, el parámetro \textbf{+exhaustive-search-strategy+} tiene como valor asociado una instancia de \textbf{exhaustive-neighborhood-search-strategy}.

A continuación se define una lista de funciones de exploración para los criterios \textit{mover un cliente dentro de su ruta}, \textit{mover un cliente a cualquier ruta en cualquier posición} e \textit{intercambiar dos clientes de posición}. En todos los casos se hace una búsqueda exhaustiva con selección de mejor vecino.

\begin{lstlisting}
(setf rab (make-neighborhood-criterion 
						`((select-route r1)
						(select-client c1 from r1)
						(insert-client c1 to r1))
						+exhaustive-search-strategy+ 
						+best-improvement+))

(setf rarb (make-neighborhood-criterion 
						`((select-route r1)
						(select-client c1 from r1)
						(select-route r2)
						(insert-client c1 to r2))
						+exhaustive-search-strategy+ 
						+best-improvement+))

(setf rarac (make-neighborhood-criterion 
						`((select-route r1)
						(select-client c1 from r1)
						(select-route r2)
						(select-client c2 from r2)
						(swap-clients c1 c2))
						+exhaustive-search-strategy+ 
						+best-improvement+))

(setf criteria (list rab rarb rarac))

\end{lstlisting}

En este punto es posible invocar la función de metaheurística de Búsqueda de Vecindad Variable y obtener una solución.

\begin{lstlisting}
(setf result (vns-vrp-system problem criteria graph :max-iter 1000))
\end{lstlisting}

La siguiente sección muestra el código completo que soluciona un CVRP con datos ficticios.

\section{Código completo}\label{4-union}

\begin{lstlisting}
(defparameter c1 (basic-cvrp-client 1 1))
(defparameter c2 (basic-cvrp-client 2 1))
(defparameter c3 (basic-cvrp-client 3 4))
(defparameter c4 (basic-cvrp-client 4 3))
(defparameter c5 (basic-cvrp-client 5 2))
(defparameter c6 (basic-cvrp-client 6 1))


(defparameter d0 (basic-depot))

(defparameter dist-mat #2A((0 1 2 3 4 5 6)
										(1 0 5 2 1 3 2)
										(2 5 0 2 2 2 2)
										(3 2 2 0 1 2 1)
										(4 1 2 1 0 2 3)
										(5 3 2 2 2 0 1)
										(6 2 2 1 3 1 0)))

(defparameter problem (finite-fleet-cvrp-problem 
										:id 1 
										:clients (list c1 c2 c3 c4 c5 c6)
										:depot d0 
										:distance-matrix dist-mat 
										:capacity 20))

(defparameter r1 (route-for-simulation 
									:id 1
									:vehicle (cvrp-vehicle 1 20) 
									:depot d0
									:clients (list c1 c2 c3 (clone d0))
									:previous-client (clone d0)))
(defparameter r2 (route-for-simulation 
									:id 2 
									:vehicle (cvrp-vehicle 2 20)
									:depot d0
									:clients (list c4 c5 c6 (clone d0))
									:previous-client (clone d0)))

(defparameter s1 (basic-solution
									:id 1 
									:routes (list r1 r2)))

(defparameter graph (init-graph s1))


(def-var total-distance 0 graph)
(loop for r in (routes s1) do
	(def-var route-distance 0 graph)
	(def-var route-demand (capacity (vehicle r)) graph) 
	(loop for c in (clients r) do
		(increment-distance (previous-client r) c route-distance dist-mat graph)
		(decrement-demand c route-demand graph) 
		(setf (previous-client r) c))
	(increment-value total-distance route-distance graph)
	(apply-penalty route-demand total-distance 10 graph) 
(return-value total-distance graph))


(setf rab (make-neighborhood-criterion 
								`((select-route r1)
								(select-client c1 from r1)
								(insert-client c1 to r1))
								+exhaustive-search-strategy+ 
								+best-improvement+))

(setf rarb (make-neighborhood-criterion 
								`((select-route r1)
								(select-client c1 from r1)
								(select-route r2)
								(insert-client c1 to r2))
								+exhaustive-search-strategy+ 
								+best-improvement+))

(setf rarac (make-neighborhood-criterion 
								`((select-route r1)
								(select-client c1 from r1)
								(select-route r2)
								(select-client c2 from r2)
								(swap-clients c1 c2))
								+exhaustive-search-strategy+ 
								+best-improvement+))

(setf criteria (list rab rarb rarac))

(setf result (vns-vrp-system problem criteria graph :max-iter 10000000000))
\end{lstlisting}

Para resolver otra variante de VRP utilizando el sistema que se implementó en el presente trabajo basta con definir las características del problema y el código de evaluación de la solución inicial. Además, cualquier problema que se defina puede ser explorado con cualquier criterio de vecindad sin importar la cantidad de operaciones que tenga y con cualesquiera combinaciones de estrategias de exploración y selección.

En este capítulo se explicó paso a paso el método de uso del sistema. Sin embargo, este sistema no es capaz de resolver todas las variantes de VRP existentes. El siguiente capítulo explica cómo extenderlo para resolver problemas que actualmente no se pueden representar.








