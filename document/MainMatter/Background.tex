\chapter{Preliminares}\label{chapter:REV-LL}

Este trabajo pretende implementar un sistema que resuelva instancias de VRP en cualquiera de sus variantes con el menor trabajo humano posible. En este capítulo se presentan los principales elementos de la investigación realizada para lograrlo.

Primeramente se comienza con una visión general del problema de enrutamiento de vehículos (VRP) y algunas de sus variantes con la sección \ref{2-VRPintro}. Se explica cómo describir a partir de código una solución de VRP.

 En \ref{2-Libr} se muestran las bibliotecas de clases existentes hasta el momento que pueden ser utilizadas para encontrar soluciones a instancias de VRP.
 
En \ref{2-Hector} se explica cómo crear un Árbol de vecindad a partir de una solución inicial y un criterio de vecindad. Este árbol es utilizado para obtener la cardinalidad de vecindades y realizar una exploración de dos fases que utiliza técnincas estadísticas.

En \ref{2-JJ} se expone el concepto de Grafo de evaluación y cómo es esto utilizado para evaluar soluciones de forma eficiente y automática.

En \ref{2-Heidy} se presenta un mecanismo para explorar vecindades de forma automática a partir de combinaciones de cualesquiera estrategias de exploración y selección.

Finalmente en \ref{2-Lisp} se describe brevemente algunas características y funcionalidades del lenguaje Common Lisp que resultaron especialmente útiles para el desarrollo del sistema.


%\newpage

\section{Problema de Enrutamiento de Vehículos}\label{2-VRPintro}
La primera referencia al VRP fue hecha por Dantzing y Ramser en \cite{Ramsin1959} en el año 1959. Se propone una formulación matemática, una aproximación algorítmica y se describe una aplicación real entregando gasolina a varias estaciones de servicio. 

En su versión más simple, el problema consta de una flota de vehículos que salen de un depósito y deben satisfacer las demandas de una serie de clientes. El objetivo es encontrar una distribución de caminos a asignar a los vehículos de forma que se optimice determinada métrica (tiempo, combustible, etc). Con más de 50 años de estudios se ha ramificado en una inmensa cantidad de variantes entre las que se pueden contar las siguientes:

\begin{itemize}
	\item CVRP - VRP con restricciones de capacidad. Cada vehículo tiene una capacidad que no debe ser excedida.
	\item VRPTW - VRP con ventanas de tiempo. Cada cliente posee un período de tiempo fijo durante el cual puede ser atendido.
	\item VRPPD - VRP con recogida y entrega. Los bienes deben ser entregados y recogidos en cantidades fijas.
	\item MDVRP - VRP con múltiples depósitos. Se cuenta con múltiples depósitos desde los que pueden salir los vehículos.
\end{itemize}

Esta es una familia de problemas NP-Duros, por lo las soluciones exactas no son factibles para instancias de grandes tamaños. Para buscar aproximaciones a la solución se utilizan heurísticas y metaheurísticas. Se destaca la búsqueda local como metaheurística que ha dado muy buenos resultados y es la seleccionada en el presente trabajo como se explica en \ref{2-Local}.

\subsection{Representación de soluciones del VRP}\label{2-Sol}
Las soluciones son representadas (en su versión más simple) como una serie de listas de clientes denominadas rutas.
Si se define a $P_1$ como un problema clásico que consta de 6 clientes: $[c_1, c_2, c_3, c_4, c_5, c_6]$, entonces una solución $s_1$ se puede definir como:

\begin{equation}
s_1 = [(c_2,c_3), (c_1,c_4,c_5), (c_6)]
\end{equation}

En $s_1$ se representa una solución con tres rutas. El vehículo perteneciente a la primera ($r_1$) ruta visita a $c_2$ y $c_3$, el vehículo de la segunda ruta ($r_2$) visita a $c_1$, $c_4$ y $c_5$ y el de la tercera ($r_3$) sólo visita a $c_6$.


\subsection{Metaheurísticas de búsqueda local}\label{2-Local}
Los algoritmos basados en búsquda local son aquellos en que se define una solución inicial y a partir de determinado criterio de vecindad se busca la solución óptima iterando por los vecinos de la vecindad formada por dicho criterio. A continuación se muestran algunos ejemplos de criterios de vecindad:

\begin{enumerate}
	\item Cambiar de posición a un cliente dentro de su ruta.
	\item Mover a un cliente de ruta.
	\item Intercambiar dos clientes de posición.
	\item Cambiar vehículo de ruta.
	\item intercambiar dos subrutas entre sí.
	\item invertir orden de una subruta. 
\end{enumerate}

Los criterios de vecindad dependen también de la variante del problema sobre la que se trabaje. Por ejemplo, el criterio de \textbf{"Cambiar vehículo"} no tiene sentido para el problema $P_1$ pues en este todos los vehículos son iguales.

Estas operaciones pueden ser obtenidas a partir de un subconjunto de operaciones más simples a las que se denomina operaciones elementales. Entre las operaciones elementales se encuentran: \textbf{selección de ruta}, \textbf{selección de cliente} e \textbf{inserción de cliente}.

Por ejemplo, \textbf{intercambiar dos clientes de posición} puede ser realizado a partir de dos \textbf{selección de ruta}, dos \textbf{selección de cliente} y dos \textbf{inserción de cliente}.

La importancia de las operaciones elementales para la implementación del sistema se explicarán en con más profundidad en \ref{label}.

A partir de un criterio y una solución inicial se pueden generar nuevas soluciones. El proceso de obtención y selección de nuevas soluciones se denomina exploración de la vecindad. Mientras más grande la vecindad es más probable encontrar mejores soluciones, pero esto puede requerir una gran cantidad de tiempo. Por tanto, las estrategias a seguir durante la exploración son un factor vital a tener en cuenta para la implementación del sistema y se explicarán en \ref{2-Hector} y \ref{2-Heidy}. Además, a la hora de comparar el costo entre dos soluciones es necesario evaluarlas, lo cual posee también un costo computacional considerable. Para la evaluación de soluciones se tiene el Grafo de evaluación explicado en \ref{2-JJ}.

\section{Vías de solución existentes}\label{2-Libr}
Una buena herramienta para resolver problemas de VRP en la actualidad es la biblioteca OR-Tools (Google Optimization Tools). Un software de código abierto útil para problemas de optimización combinatoria entre los que se encuentra el Problema de Enrutamiento.

NOTA PARA FERNANDO: Profe, no sé qué más poner aquí XD.

\section{Árbol de vecindad y exploración de dos fases}\label{2-Hector}
La exploración de vecindades puede ser ineficiente y difícil de programar. La idea que se propone es utilizar técnicas estadísticas para analizar cuáles son las mejores regiones de las vecindades para intensificar la búsqueda en estas.

Buscando aplicar dichas técnicas estadísticas es necesario saber la cardinalidad de las vecindades y separarlas en regiones. Para lograr esto de forma eficiente (sin iterar por todos los elementos de una vecindad) se propone la creación de un Árbol de vecindad. El Árbol de vecindad utiliza oun concepro de $solucion$ diferente al explicado en \ref{2-Sol}. Previamente se definió una solución de VRP en su forma más trivial como una lista de caminos conformados a su vez por la lista de clientes que se visitan en cada uno, por ejemplo:

\begin{equation}
s_1 = [(c_2,c_3), (c_1,c_4,c_5), (c_6)]
\end{equation}

Con el propósito de contar la cantidad de soluciones que tiene una vecindad, pierde importancia saber qué clientes se visita en cada ruta, en cambio sólo es necesario conocer la cantidad de clientes visitados en cada una. A este tipo de solución se le denomina solución de conteo y aplicado a $s_1$ daría como resultado:

\begin{equation}
sc_1 = [2, 3, 1]
\end{equation}

Teniendo una solución inicial, todas las soluciones de una vecindad pueden ser obtenidas aplicando sobre esta el criterio de vecindad en cuestión, con todos sus valores posibles. A la asignación de valores de un criterio de vecindad sobre una solución se le llamará una instanciación de dicho criterio. Por ejemplo, dado el criterio \textbf{mover cliente} dado por: 

\begin{itemize}
	\item Seleccionar ruta (r).
	\item Seleccionar cliente en ruta (a).
	\item Seleccionar ruta (r).
	\item Insertar cliente en ruta (b).
\end{itemize}
Las letras $r$, $a$ y $b$ son los símbolos representativos de cada operación.

Un ejemplo de instanciación de este criterio sobre la solución $s_1$ se presenta como:

\begin{itemize}
	\item $r_1 = 1$ $\rightarrow$ $r_1$ es la ruta de la que el cliente $c_1$ es extraído.
	\item $c_1$.position = 1 $\rightarrow$ $c_1$ es el primer cliente de la ruta.
	\item $r_2 = 1$ $\rightarrow$ $r_2$ es la ruta de la que el cliente $c_1$ será insertado.
	\item $c_1$.position = 2 $\rightarrow$ $c_1$ es insertado en la posición 2 de la ruta seleccionada.
\end{itemize}

A cada vecino de un criterio se le puede hacer corresponder una instanciación del mismo y por tanto, encontrar la cardinalidad de una vecindad es similar a encontrar el número de instanciaciones posibles del criterio asociado.

Al generar las distintas instanciaciones de un criterio de vecindad para
una solución dada se cumple que muchas de estas comparten una secuencia común de operaciones instanciadas. La estrategia propuesta se basa en
agrupar aquellos criterios instanciados para los cuales dicha secuencia común comience en la primera operación de los mismos, pues de esta forma
el resto de las operaciones de tales criterios instanciados no se ven afectadas, y contar para cada una de estas el número de posibles secuencias de operaciones instanciadas que unidas con la secuencia común forman un
criterio instanciado.

Al computar la cardinalidad de una vecindad del VRP, se utiliza una
estrategia recursiva que consiste en contar para una operación el número
de secuencias de operaciones instanciadas que se pueden formar con el
resto de las operaciones del criterio, así como combinar las mismas con
las posibles instanciaciones de la operación actual aprovechando que para
las operaciones modificadoras estas secuencias son comunes a todas las
posibles instanciaciones de las mismas.

Es posible utilizar este algoritmo, para almacenar toda la información
necesaria para cualquier procesamiento posterior sobre dicha vecindad en
una estructura arbórea que será llamada árbol de vecindad y que constituye
una representación de la vecindad en cuestión. En \ref{fig:neigh-tree} se muestra una representación del árbol de vecindad asociado a l criterio \textbf{mover cliente} ($rarb$). Cada nodo del árbol, representa una
operación de vecindad y almacena toda la información necesaria para instanciar dicha operación y de esta forma, a partir del proceso que computa
la cardinalidad de la vecindad, se puede pasar a generar todas las soluciones de la misma.


% TODO: \usepackage{graphicx} required
\begin{figure}
	\centering
	\includegraphics[width=0.9\linewidth]{Graphics/Neigh-Tree}
	\caption{Árbol de vecindad asociado a $rarb$}
	\label{fig:neigh-tree}
\end{figure}

Cada rama del árbol representa un conjunto de soluciones, por tanto, el conjunto de ramas representa una partición de la vecindad a la que está asociado dicho árbol. A los conjuntos hechos por esta partición se les denomina regiones y resultan útiles para encontrar características compartidas en grupos de vecindades. Por ejemplo, es conveniente analizar cuáles son las regiones con mejores soluciones para intensificar en estas la búsqueda de soluciones óptimas.

Luego de generadas las soluciones a partir del Árbol de Vecindad, es necesario conocer sus costos. Para esto se utiliza un Grafo de Evaluación.

\section{Grafo de Evaluación}\label{2-JJ}
Durante la exploración de las vecindades es (generalmente) necesario evaluar las soluciones que se van analizando. Ya sea para retornar inmediatamente una solución mejor a la inicial (estrategia de selección de primera mejora), para devolver la mejor solución (estrategia de selección de mejor vecino) o un vecino aleatorio entre todos los que mejoren la solución (estrategia de selección de mejor vecino aleatorio), en todos los casos hay que determinar el costo de las soluciones exploradas para analizar lo que es "mejor". Encontrar el costo de una solución es lo que se denomina como evaluar.

La evaluación de soluciones es potencialmente costosa. En el caso más simple se debe sumar las distancias entre cada cliente de cada ruta y el depósito. Agregar restricciones implica análisis extra como la aplicación de penalizaciones a rutas con vehículos sobrecargados en el caso de CVRP.

El Grafo de Evaluación, propuesto por Jose Jorge Rodríguez en \ref{2-JJ}, permite evaluar soluciones vecinas a una solución inicial de forma eficiente ya que garantiza que sólo se recalculan los fragmentos de las nuevas soluciones en los que estas se diferencien de la solución inicial.

La estructura propuesta es una representación en forma de grafo de la evaluación de la función objetivo en una determinada solución. Sus nodos están divididos en dos tipos: Nodos de alto nivel y nodos de bajo nivel. Los nodos de bajo nivel representan operaciones (incremento, decremento) que reciben como entradas ciertos nodos de alto nivel y modifican con sus salidas otros nodos de alto nivel.

Por ejemplo, en \ref{fig:eval-graph-1} se muestran los nodos de alto nivel del grafo que representa la solución:

\begin{equation}
s = [(c_1,c_2), (c_3,c_4)]
\end{equation}

Nótese que todas las rutas en el grafo comienzan y terminan con el nodo que representa al depósito. El nodo $cost$ almacena el valor del costo total de la solución.

\begin{figure}
	\centering
	\includegraphics[width=0.9\linewidth]{Graphics/eval-graph-1}
	\caption{Nodos de alto nivel en un grafo de evaluación que representa la solución $s1$ de VRP clásico.}
	\label{fig:eval-graph-1}
\end{figure}

En \ref{fig:eval-graph-2} se muestra una representación del grafo completo para esta solución. Los nodos con símbolo de incremento son nodos de bajo nivel que toman como entrada dos nodos clientes (o depósito) y como salida adicionan al nodo de costo total la distancia entre ellos.

\begin{figure}
	\centering
	\includegraphics[width=0.9\linewidth]{Graphics/eval-graph-2}
	\caption{Grafo de evaluación que representa la solución $s1$ de VRP clásico.}
	\label{fig:eval-graph-2}
\end{figure}

En \ref{fig:eval-graph-3} se transforma el problema en CVRP utilizando la misma solución. En este caso se agregan los nodos de alto nivel $cap$ que tienen como valor inicial la capacidad del vehículo perteneciente a cada ruta. Los nodos de decremento reciben un cliente como entrada y, como salida, disminuyen la capacidad del vehículo en una cantidad igual a la demanda del cliente. Luego, los nodos $pen$ (también de bajo nivel) reciben como entrada los nodos de capacidad y, en caso de tener estos valor negativo, como salida penalizan el costo total de la solución.

 \begin{figure}
	\centering
	\includegraphics[width=0.9\linewidth]{Graphics/eval-graph-3}
	\caption{Grafo de evaluación que representa la solución $s1$ de VRP con restricciones de capacidad.}
	\label{fig:eval-graph-3}
\end{figure}

Todos los nodos de bajo nivel tienen asociado una función \textbf{evaluate} (para evaluar) y una función \textbf{undo} (para desevaluar) que son ejecutados cuando se agregan o remueven nodos de alto nivel. Por ejemplo, retirar un cliente $C1$ del grafo representado en \ref{fig:eval-graph-2} (VRP clásico) provoca que los dos nodos de incremento que utilizan dicho cliente como entrada se desevalúen y al mismo tiempo se crea un nuevo nodo incremento que recibe como entrada $d$ y $c2$. Luego, insertar a $c1$ al final de la ruta $r2$ implicaría desevaluar el nodo incremento que recibe como entrada a $c4$ y $d$ mientras se crean dos nodos incrementos nuevos, uno recibiendo de entrada a $c4$ y $c1$ mientras que el otro a $c1$ y $d$. El resultado de estas dos operaciones se muestra en \ref{fig:eval-graph-4} y es, precisamente, el grafo resultante de aplicar la siguiente instanciación del criterio $rarb$:

\begin{itemize}
	\item Seleccionar ruta (r1).
	\item Seleccionar cliente (c1) en ruta (r1).
	\item Seleccionar ruta (r2).
	\item Insertar cliente (c1) en posición (3) en ruta (r2).
\end{itemize}

Nótese que luego de aplicar los métodos \textbf{evaluate} y \textbf{undo} correspondientes el nodo $cost$ tiene almacenado el costo de la solución resultante luego de aplicar una instancia del criterio $rarb$. Para encontrar el costo de la nueva solución sólo fue necesario analizar y modificar los nodos en que el grafo de la solución nueva se diferencia con la solución anterior y no todo el grafo. En esto se basa la evaluación "eficiente" del Grafo de Evaluación.

 \begin{figure}
	\centering
	\includegraphics[width=0.9\linewidth]{Graphics/eval-graph-4}
	\caption{Grafo de evaluación que representa la solución $s1$ de VRP clásico luego de aplicada una instancia de $rarb$}
	\label{fig:eval-graph-4}
\end{figure}

Para construir el grafo que representa la solución inicial el usuario debe ingresar el códio de evaluación de esa primera solución. Luego, los costos el resto de las soluciones generadas son obtenidos al aplicar operaciones sobre el grafo inicializado.

Para explorar una vecindad a partir de un criterio es necesario (además de generar y evaluar soluciones) decidir e implementar estrategias de exploración y de selección. A partir de combinaciones de diferentes estrategias se pueden realizar montones de exploraciones distintas.


\section{Combinación de estrategias de exploración y selección.}\label{2-Heidy} 
En el proceso de exploración de una vecindad se parte de una solución inicial, se generan soluciones vecinas a esta (con el Árbol de vecindad) y se evalúan para comparar y obtener mejores soluciones (Grafo de Evaluación). Al explorar, deben también tenerse en cuenta problemas tales como la cardinalidad potencialmente enorme de los vecinos o el retorno de mínimos locales. Para una vecindad muy poblada tal vez dé mejor resultado explorar no todos sus vecinos, sino una porción aleatoria de estos. Tal vez retornar siempre al mejor vecino de cada vecindad pueda provocar el encuentro de un mínimo local que se hubiera evitado seleccionando aleatoriamente cualquier solución que mejorara la inicial.

Al espectro de búsqueda de vecinos en una vecindad se le denomina estrategia de exploración. Algunos ejemplos son:

\begin{itemize}
	\item Exploración exhaustiva: Se analizan todos los vecinos que puedan ser generados.
	\item Exploración aleatoria: Se genera una cantidad fija de vecinos menor que la cardinalidad de la vecindad. La decisión de qué vecinos generar es aleatoria.
\end{itemize}

Además de decidir qué vecinos explorar, también es necesario decidir cuál retornar entre aquellos que mejoran la solución. A esta decisión se le denomina estrategia de selección. Se tiene como ejemplos:

\begin{itemize}
	\item Mejor vecino: Se retorna al mejor vecino entre todos aquellos analizados.
	\item Primera mejora: En el momento en que se encuentra un vecino mejor que el inicial, este es retornado.
	\item Mejora aleatoria: Se retorna un vecino aleatorio entre todos aquellos mejores que la solución inicial.
\end{itemize}

A partir de distintas combinaciones de estrategias de exploración y selección es posible realizar numerosas exploraciones distintas que obtengan diferentes resultados.

La propuesta de Heidy Abreu en \cite{Heidy} permite generar funciones de exploración fabricadas a partir de un criterio, una estrategia de selección y una de exploración. Las estrategias se definen como clases que se pasan como instancias a la función generadora. La función resultante recibe el problema con una solución inicial, ejecuta la exploración y retorna una solución mejor en caso de existir dentro de la vecindad definida por el criterio. En este trabajo, el árbol de vecindad y el grafo de evaluación forman parte también de las funciones de exploración creadas. Un árbol de vecinad genera las soluciones y un grafo de evaluación de usa para evaluarlas. El grafo debe ser pasado como entrada.

La creación automática de funciones de exploración con distintas combinaciones de estrategias de exploración y selección se logra aprovechando sus características comunes y estructuras similares. El código generado en todas las funciones se reparte en cinco regiones que se  unen para generar una función completa. Las regiones y el tipo de código que pertenece a cada una se explicarán en \ref{TODO}.

El trabajo de generación de código y la creación de funciones a partir de estrategias diferentes se facilita mucho gracias a las características del lenguaje Common Lisp.

\section{Common Lisp y sus funcionalidades}\label{2-Lisp}
Common Lisp es un lenguaje de programación multi paradigma (soporta una combinación de paradigmas de programación tales como la programación imperativa, funcional y orientada a objetos). Facilita el desarrollo de software evolutivo e incremental, con la compilación iterativa de programas eficientes en tiempo de ejecución.

El lenguaje acepta herencia múltiple. Esta característica permite crear jerarquías con clases que implementan funcionalidades de varias clases superiores. Por ejemplo, la clase que representa la estrategia de selección de mejor vecino (\textbf{best-improvement}) hereda de una clase que indica  retorno de mejor solución (\textbf{return-best-solution}) y de otra que indica el uso de un grafo de evaluación (\textbf{use-eval-graph}).

La herencia útil resulta especialmente útil para el presente trabajo cuando se combina con el sistema CLOS (Common Lisp Object System). Un mismo método puede tener numerosas implementaciones (especificaciones) que se ejecutan de acuerdo a los tipos de los parámetros proveídos como entradas. Además, también se ejecutan todas las especificaciones cuyos parámetros coincidan con los ancestros de los tipos de las entradas. Todas las especificaciones se combinan alrededor de un método primario conformando un único método con la unión de todos los códigos. El orden en que se combinan los métodos depende del orden de herencia y de los parámetros :before, :around y :after con que se definen.

Como ejemplo se tiene que para generar el código de la función de exploración para la estrategia de selección de menor vecino se une el código de métodos cuyo parámetro de \text{search-strategy} tenga tipo \textbf{best-improvement}, \textbf{return-best-solution}, \textbf{use-eval-graph} y cualquier otra clase que herede de alguna de estas.



























