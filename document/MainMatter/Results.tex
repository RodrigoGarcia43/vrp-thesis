\chapter{Experimentos y resultados}\label{chapter:Results}

En este capítulo se presentan las pruebas que se hicieron para evaluar el desempeño del presente sistema. Para esto pruebas se utilizaron datos de CVRP reales conocidos como \textit{A-n33-k5} \cite{TODO}. Se utilizó el algoritmo de búsqueda local \textit{Búsqueda de Vecindad Variable} \cite{TODO} con funciones de exploración generadas a partir \textbf{+exhaustive-search-strategy+}, \textbf{+best-solution-selection-strategy+} y los siguientes criterios de vecindad:

\begin{itemize}
	\item rab.
	\item rarb.
	\item rarac.
	\item ref.
	\item rerf.
	\item rereg.
\end{itemize}


\textit{A-n33-k5} tiene 32 clientes y se conoce una solución óptima de costo 661.

Luego de aplicar la metaheurística se obtuvo la siguiente solución con costo 675, muy cercano al costo óptimo.

\begin{lstlisting}
S1: (675)
routes:
<r5: <cv:1. 0/100> (<d:0>: (<c2: 23> <c24: 13> <c6: 18> <c23: 14>))>
<r4: <cv:1. 0/100> (<d:0>: (<c20: 8> <c32: 3> <c13: 23> <c8: 10> <c7: 19>
<c26: 2> <c4: 13> <c22: 19>))>
<r3: <cv:1. 0/100> (<d:0>: (<c15: 18> <c17: 24> <c9: 18> <c3: 14> <c16: 10>
<c29: 8>))>
<r2: <cv:1. 0/100> (<d:0>: (<c28: 15> <c18: 13> <c31: 24> <c1: 5> <c21: 10>
<c14: 9> <c19: 14> <c11: 5>))>
<r1: <cv:1. 0/100> (<d:0>: (<c12: 9> <c5: 8> <c27: 23> <c25: 14> <c30: 20>
<c10: 20>))>
\end{lstlisting}

Luego, para comprobar las capacidades del sistema para resolver distintas variantes con trabajo humano mínimo se le agregó una restricción al problema anterior. Según la nueva restricción, cada una de las rutas de la solución tiene un límite máximo de 7 clientes. 

El código que se necesitó para resolver el nuevo problema se muestra a continuación. Primero se definen nuevas clases para los nodos que se agregan al grafo.

\begin{lstlisting}
(def-vrp-class limited-client (demand-client)())
    
(def-vrp-class size-limited-cvrp-client (limited-client) ()
	:constructor (size-limited-cvrp-client (id demand)))
	
(def-vrp-class input-limited-node (input-node) 
	((limit-calculator :initform nil))
	:constructor (new-limited-node (&key content)))

 (def-vrp-class input-limited-distance-demand-node (input-distance-node input-demand-node input-limited-node) ()
	:constructor (new-input-limited-distance-demand-node (&key content)))
	
 (def-vrp-class decrement-size-node (low-level-node)
	((input-with-limit))
	:constructor (new-decrement-size-node (&key previous-node )
\end{lstlisting}

Luego se implementan los métodos de construcción y modificación del grafo.

\begin{lstlisting}
 (defmethod convert-to-node :around ((target limited-client) graph)
	(let ((new-c (new-input-limited-distance-demand-node :content target)))
		(progn
			(setf (inputs graph) (append (inputs graph) `(,new-c)))
			(setf (gethash target (class-to-io graph)) new-c))))



(defmethod evaluate-low-level-node ((ll-node decrement-size-node))
	(progn
		(if (not (typep (content (input-with-limit ll-node)) 'basic-depot))
			(decf (output-value (output-action ll-node)) 1))
		(if (updater (output-action ll-node))
			(undo-low-level-node (updater (output-action ll-node))))))


(defmethod undo-low-level-node ((ll-node decrement-size-node))
	(progn
		(if (not (typep (content (input-with-limit ll-node)) 'basic-depot))
			(incf (output-value (output-action ll-node)) 1))
			(undo-low-level-node (updater (output-action ll-node)))))


(defmethod remove-node append ((t-node input-limited-node))
	(if (not (typep t-node 'input-depot-node))
		(progn
			(undo-low-level-node (limit-calculator t-node))))))
			

(defmethod insert-node append ((t-node input-limited-node) 
(i-node input-limited-node))
	(let* ((new-inc (new-decrement-size-node 
								:output-action (output-action (limit-calculator t-node))
								:input-with-limit i-node)))
	(progn
		(setf (limit-calculator i-node) new-inc)
		(evaluate-low-level-node new-inc))))

		
(defmethod decrement-size (c capacity-slot-accumulator graph)
	(let* ((l-node (gethash c (class-to-io graph)))
				(acc (gethash capacity-slot-accumulator (slot-to-output graph)))
				(l-calc (new-decrement-size-node :output-action acc :input-with-limit l-node)))
		(progn
			(setf (limit-calculator l-node) l-calc)
			(evaluate-low-level-node l-calc))))
\end{lstlisting}

En este punto sólo resta definir el código de evaluación de una solución para el nuevo problema.

\begin{lstlisting}

(progn
	(def-var total-distance 0 graph)
	(loop for r in (routes s1) do 
		(progn
			(def-var route-distance 0 graph)
			(def-var route-demand (capacity (vehicle r)) graph) 
			(def-var route-limit 7 graph)
			(loop for c in (clients r) do 
				(progn
					(increment-distance (previous-client r) c route-distance A-n33-k5-distance-matrix graph)
					(decrement-demand c route-demand graph) 
					(decrement-size c 'route-limit graph) 
					(setf (previous-client r) c)))
			(increment-value total-distance route-distance graph)
			(apply-penalty route-demand total-distance 100 graph) 
			(apply-penalty route-limit total-distance 1000 graph))) 
	(return-value total-distance graph)))
\end{lstlisting}

Luego de ejecutar el sistema con los mismos datos de \textit{A-n33-k5} para el nuevo problema, se obtuvo la siguiente solución de costo 727:

\begin{lstlisting}
S1: (727)
routes:
<r5: <cv:1. 0/100> (<d:0>: (<c22: 19> <c15: 18> <c16: 10> <c3: 14> <c9: 18>
<c4: 13> <c20: 8>))>
<r4: <cv:1. 0/100> (<d:0>: (<c2: 23> <c32: 3> <c13: 23> <c8: 10> <c7: 19>
<c26: 2> <c10: 20>))>
<r3: <cv:1. 0/100> (<d:0>: (<c23: 14> <c28: 15> <c18: 13> <c11: 5> <c6: 18>
<c24: 13>))>
<r2: <cv:1. 0/100> (<d:0>: (<c29: 8> <c31: 24> <c1: 5> <c21: 10> <c14: 9>
<c19: 14>))>
<r1: <cv:1. 0/100> (<d:0>: (<c12: 9> <c5: 8> <c27: 23> <c25: 14> <c30: 20>
\end{lstlisting}

El siguiente capítulo presenta las conclusiones y recomendaciones.