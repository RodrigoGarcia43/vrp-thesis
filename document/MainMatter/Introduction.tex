%===================================================================================
% Chapter: Introduction
%===================================================================================
\chapter*{Introducción}\label{chapter:introduction}
\addcontentsline{toc}{chapter}{Introducción}
%===================================================================================

\qquad 
Los costos por transportación pueden representar más de la mitad de los costos logísticos de una empresa \cite{factoid}. Esto implica que para empresas con capitales millonarios, una buena planeación de sus rutas de transportación puede representar un ahorro igualmente millonario.

El problema de enrutamiento de vehículos (VRP por sus siglas en inglés) es un problema de optimización combinatoria cuyo objetivo en su forma más simple es, dado un conjunto de clientes, un depósito y una flota de vehículos, encontrar una asignación de rutas que optimice ciertos criterios tales como tiempo y costos de transportación.

En la literatura se reportan múltiples variantes para este problema como el Problema de Enrutamiento de Vehículos con restricciones de capacidad (CVRP) \cite{CVRP}, el Problema de Enrutamiento de Vehículos con ventanas de tiempo y el Problema de Enrutamiento de Vehículos con recogida y entrega (VRPPD) \cite{2002vehicle}, por citar sólo algunos de las más estudiadas. Cada variante tiene restricciones propias, por lo que resulta difícil la creación de un método de solución universal para cualquier VRP.

Estos problemas son NP-duros, esto implica que no existen soluciones exactas para conjuntos de datos reales y por tanto, se utilizan técnicas aproximadas como heurísticas y metaheurísticas que han sido objeto de estudio por décadas \cite{1964scheduling,1974networks,1981complexity,1997tabu,2002vehicle}.

En particular, los algoritmos de búsqueda local han mostrado buenos resultados en los últimos tiempos \cite{TODO}.

El proceso de solución de un Problema de Enrutamiento de Vehículos específico es complejo y su solución puede tomar tiempo. Por ejemplo, varios meses en el caso de un trabajo de diploma \cite{TODO}, hasta un año en caso de una maestría \cite{Alina2010} y varios años en caso de una tesis de doctorado \cite{Alina2017}.

%Cabe destacar la biblioteca OR-Tools, un software de código abierto útil para resolver problemas de optimización combinatoria entre los que se encuentra VRP \cite{TODO}.

Desde hace varios años, comenzando por \cite{Camila}, en la facultad de Matemática y Computación de La Universidad de La Habana se han desarrollado varias investigaciones con el objetivo de agilizar el tiempo que toma resolver una variante específica de VRP. Para ello se propone utilizar una metaheurística de búsqueda local (IVNS) que permite considerar infinitos criterios de vecindad para un problema, a partir de gramáticas libres del contexto. Luego, en \cite{Daniela} se propone una forma de obtener automáticamente, a partir de la descripción del problema, la gramática necesitada por IVNS. En \cite{JJ} se propone una manera de calcular el costo de un vecino en casi cualquier variante de VRP y para ello se utiliza un Grafo de Evaluación. En \cite{Hector} se plantean formas automáticas para explorar vecindades de distintas maneras mediante un Árbol de Vecindad. Finalmente, en \cite{Heidy} se propone una vía para combinar estrategias de exploración y selección sin tener que programar todas las combinaciones posibles a través de un generador de funciones de exploración. Cada una de estas herramientas resuelve un problema por separado, pero hasta ahora no era posible su integración. Si se lograran integrar sería posible resolver cualquier problema de VRP utilizando cualquier combinación de estrategias de exploración y criterios de vecindad a partir de \cite{Hector} y \cite{Hector}. Además usando infinitos criterios de vecindad al combinar lo propuesto en \cite{Camila} y \cite{Daniela}. La única restricción estaría en que los vecinos fueran calculables por lo propuesto en \cite{JJ}.

En este trabajo se propone una herramienta que permite integrar el Grafo de Evaluación de \cite{JJ}, con el Árbol de Vecindad de \cite{Hector} y el generador de funciones de exploración de \cite{Heidy}. Por ahora no es objetivo integrar el sistema con IVNS y por tanto se proponen otras metaheurísticas de búsqueda local como VNS \cite{TODO}.

Con el sistema propuesto, resolver una variante de VRP se traduce en definir cómo se evalúa una única solución y definir, en un lenguaje casi natural, qué estrategias de exploración y criterios de vecindad se quieren utilizar.

\subsection*{Objetivos}
El objetivo general de este trabajo es diseñar e implementar un sistema que permita calcular los costos de cualquier vecino usando el Grafo de Evaluación de \cite{JJ}, que las vecindades se exploren usando el Árbol de Vecindad de \cite{Hector}, que las estrategias de exploración y selección se puedan combinar con lo propuesto en \cite{Heidy} y, por tanto, resolver variantes de VRP a partir únicamente del código de evaluación de una solución.

Para lograr los objetivos generales se han trazado los siguientes objetivos específicos:

\begin{enumerate}
	\item Consultar literatura especializada sobre el estado del arte de los problemas VRP.
	\item Estudiar las ideas y códigos relacionados con el Grafo de Evaluación \cite{JJ}, el Árbol de Vecindad \cite{Heidy}, y el generador de funciones de exploración \cite{Hector}.
	\item Diseñar e implementar un sistema que combine estas ideas para resolver distintas variantes de VRP a partir de la evaluación de una solución.
	\item Analizar los resultados obtenidos. Para ello se utiliza el sistema implementado para resolver, con muy poco esfuerzo humano variantes de VRP.
\end{enumerate}
El presente documento está organizado en 5 capítulos.

TODO:
En el capítulo \ref{Chapter: Tools} se describen los elementos fundamentales del Problemas de Enrutamiento de Vehículos. En el capítulo \ref{Chapter: Tools} se describen las de las tesis que se combinan en este trabajo \cite{JJ,Heidy,Hector}. El capítulo \ref{chapter:Solution} describe el sistema que se implementó. En el capítulo \ref{chapter:Tutorial} se describe el método de uso el sistema para describir y resolver variantes de VRP. En el capítulo \ref{chapter:Extension} se muestra cómo extender este sistema. En el capítulo \ref{chapter:Results} se exponen los resultados alcanzados. En el capítulo \ref{chapter:conclusions} se ofrecen las conclusiones y se brindan algunas recomendaciones para trabajos futuros.





