\begin{opinion}	
	%El Problema de Enrutamiento de Vehículos (VRP) es uno de los problemas más estudiado dentro de la logística, por su importancia práctica y académica.  En nuestra facultad se han propuesto varias vías de solución para este tipo de problema, pero siempre usando algoritmos de optimización discreta.  Por otro lado, los algoritmos de optimización continua suelen tener mejores resultados y ser más rápidos que los discretos, ventajas estas que estaban vedadas al VRP...  hasta ahora.
	
	Con este trabajo se explora la posibilidad de convertir el espacio de soluciones discreto de un VRP a un espacio continuo sobre el que se puede aplicar cualquiera de los algoritmos de optimización continua existentes.  Para realizar esta transformación se usan técnicas de aprendizaje de máquinas.
	
	Creo que la palabra clave en este trabajo es aprendizaje, y no solo el que realizan los \textit{autoencoders} para convertir elementos de un espacio en elementos de otro, sino el que hemos realizado todos los involucrados en este proyecto.
	
	Para realizar esta tesis Dalianys tuvo que incursionar en (aprender) temas que no están incluidos en su plan de estudios, (aprender a) resolver creativamente problemas que aparecieron como parte del proceso de desarrollo y adquirir habilidades (aprender) relacionadas con la escritura de documentos científicos.
	
	Por otra parte, los demás involucrados también hemos aprendido: de la constancia y dedicación de Dalianys, de su independencia, y de su capacidad de búsqueda, organización, procesamiento y análisis de información científica.  Por todo eso, y por la trayectoria que ha tenido como estudiante y como alumna ayudante, considero que estamos en presencia de un excelente trabajo, que resume una excelente trayectoria, de una excelente profesional de la Ciencia de la Computación.
	
	\vspace{1cm}
	
	
	\begin{flushright}
		\underline{\hspace{6.5cm}}\\
		MSc. Fernando Raul Rodriguez Flores
		
		Facultad de Matemática y Computación
		
		Universidad de la Habana
		
		Noviembre, 2021
	\end{flushright}

\end{opinion}