\begin{abstract}
El problema de enrutamiento de vehículos (VRP) es uno de los problemas más estudiados en el área de la optimización combinatoria. Su objetivo es diseñar un sistema de rutas que permita satisfacer las demandas de un conjunto de clientes de manera eficiente. Una solución al problema se puede interpretar como una lista de rutas donde cada una de estas constituye una secuencia de los clientes que se visitarán en la misma. Este tipo de representación discreta de las soluciones, ha sido empleada por algoritmos heurísticos y metaheurísticos para resolver el problema. Sin embargo, en este trabajo, se presenta un modelo auxiliado de técnicas de aprendizaje automático para obtener una representación continua de las soluciones.

Se propone una herramienta para transformar soluciones del VRP a vectores reales basada en la teoría de los modelos de redes neuronales que se conocen como \textit{autoencoders}. En particular, se formularon dos propuestas concretas: el modelo \textit{LinearAEC}, fundamentado en un \textit{autoencoder} clásico, y el modelo \textit{VAE}, basado en los \textit{variational autoencoders}. El comportamiento de ambos modelos fue similar, mostrando una alta capacidad de generación de soluciones válidas al problema de enrutamiento. 


%Gracias a ello es posible conformar el espacio de soluciones continuas a partir de las codificaciones obtenidas. 

%Se realiza una introducción al área del aprendizaje automático 


%La aplicación de redes neuronales para optimizar decisiones en problemas de optimización combinatoria se remonta a \textit{Hopfield et al.} \cite{Hopfield1982}; llegando a resolver pequeñas instancias del TSP de hasta 30 ciudades. El desarrollo de arquitecturas de redes neuronales ha allanado el camino a enfoques competitivos basados en ML no solo para dar solución a problemas de enrutamiento, sino también para inspeccionar el espacio de búsqueda continuo que conforman un conjunto de soluciones.

\end{abstract}

\begin{enabstract}
The vehicle routing problem (VRP) is one of the most studied problems in the area of combinatorial optimization. Main objective is to design a routing system that allows to satisfy the demands of a set of customers in an efficient way. A solution to the problem can be interpreted as a list of routes where each route is a sequence of customers to be visited on that route. This type of discrete representation of the solutions has been used by heuristic and metaheuristic algorithms to solve the problem. However, in this work, a model aided by machine learning techniques is presented to obtain a continuous representation of the solutions.

A tool for transforming VRP solutions to real vectors based on the theory of neural network models known as autoencoders is proposed. In particular, two specific proposals were formulated: the \textit{LinearAEC} model, based on a classical autoencoder, and the \textit{VAE} model, based on variational autoencoders. The behavior of both models was similar, showing a high capacity to generate valid solutions to the routing problem.
	
\end{enabstract}