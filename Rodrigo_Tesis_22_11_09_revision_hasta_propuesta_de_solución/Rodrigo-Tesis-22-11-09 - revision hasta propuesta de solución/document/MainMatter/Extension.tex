\chapter{Extensibilidad}\label{chapter:Extension}

El sistema de solución de VRP desarrollado puede resolver de forma nativa varias de las variantes de VRP más conocidas. Sin embargo, en algún momento, un usuario potencialmente deberá resolver un problema no soportado nativamente. En este capítulo se explica cómo extender el sistema para otras variantes.

El primer paso a realizar es la extensión de las clases que describen el problema.

\section{Descripción del problema}\label{4-description}
Para extender el sistema a nuevas variantes de VRP puede ser necesario definir nuevas clases mediante las cuales sea posible su descripción. Las especificaciones de cada característica del problema depende de las clases estructurales de las que estas hereden. Además, todas las clases que representan características de problemas son descendientes de su clase base correspondiente. Por ejemplo, un cliente de CVRP (clase \textbf{basic-cvrp-client}) hereda de \textbf{demand-client}, clase que indica que el cliente posee una demanda y de \textbf{basic-client}, la clase base para los clientes.

Definir una clase nueva implica identificar de qué clases estructurales debe heredar para satisfacer sus especifiaciones y, en caso de ser necesario, crear nuevas clases estructurales. Definir un problema nuevo puede implicar la creación de varias clases para representar las características de este.

A continuación se ejemplificará cómo crear la clase \textbf{basic-time-windows-problem} que representa un Problema de Enrutamiento de Vehículos con Vetanas de tiempo (TWVRP) \cite{TODO}. En este problema los clientes, además de sus demandas, tienen un período de tiempo en que pueden ser visitados, de lo contrario la solución es penalizada. Además, los vehículos deben esperar ciertas cantidades de tiempo mientras atienden a cada cliente.

Para definir el TWVRP es necesario definir primero clientes que conozcan sus ventanas de tiempo y el tiempo que debe consumir el vehículo atendiéndolos. Se definen las clases estructurales siguientes:

\begin{itemize}
	\item \textbf{time-windows-client}: Tiene ventana de tiempo.
	\item \textbf{service-time-client}: Consume tiempo al ser atendido. 
\end{itemize}

Entonces se crea la clase \textbf{basic-tw-client} que representa un cliente de TWVRP y hereda de:

\begin{itemize}
	\item \textbf{basic-client}
	\item \textbf{demand-client}
	\item \textbf{time-windows-client}
	\item \textbf{service-time-client}
\end{itemize}

Además, se deben definir nuevas rutas que conozcan el tiempo actualmente consumido. Se crea la clase estructural \textbf{route-with-time} y la clase \textbf{basic-tw-route} que hereda de \textbf{route-with-time} y \textbf{basic-route} (o \textbf{route-for-simulation} si se quiere utilizar en el Grafo de Evaluación).

También debe definirse la clase abstracta \textbf{time-problem} para indicar que el problema tiene una matriz de $nxn$ cuyas posiciones guardan los tiempos necesarios para viajar entre clientes. Finalmente es posible definir la clase \textbf{basic-time-windows-problem} para el problema con ventanas de tiempo que hereda de las siguientes clases abstractas:

\begin{itemize}
	\item \textbf{basic-problem}
	\item \textbf{distance-problem}
	\item \textbf{capacity-problem}
	\item \textbf{time-problem}
\end{itemize}

Junto con las nuevas características, definir un nuevo problema implica también definir cómo son evaluadas sus soluciones. Diferentes evaluaciones pueden necesitar del agrego de nuevos tipos de nodos y funciones de construcción para el Grafo de Evaluación.

\section{Evaluación}\label{4-eval}

TODO: CAMBIAR EL EJEMPLO

Al definir nuevos problemas, también se debe definir la forma en que sus soluciones son evaluadas. Esto implica tener que extender el Grafo de Evaluación para satisfacer nuevas características y restricciones añadidas. Extender el grafo se traduce en agregar los nuevos nodos y funciones de construcción de grafo que sean necesarios.

Por ejemplo, el código de evaluación de una solución de TWVRP se muestra a continuación:

\begin{lstlisting}
(progn
	(def-var total-distance 0 graph)
	(loop for r in (routes s1) do 
		(progn
		(def-var route-distance 0 graph)
		(def-var route-demand (capacity (vehicle r)) graph) 
		
		(def-var route-time 0 graph)
		(def-var route-time-penalizer 0 graph)
		
		(loop for c in (clients r) do 
			(progn
			(increment-distance (previous-client r) c route-distance dist-mat graph)
			(decrement-demand c route-demand graph) 
			
			(increment-time (previous-client r) c route-time time-mat graph)
			(increment-time-penalizer c route-time route-time-penalizer)
		
		(setf (previous-client r) c)))
		(increment-value total-distance route-distance graph)
		(apply-penalty route-demand total-distance 10 graph)) 
	(apply-penalty route-time-penalizer total-distance 10 graph)
	(return-value total-distance graph)))
\end{lstlisting}

La diferencia con el código de evaluación de CVRP está en las nuevas variables que almacenan el tiempo que ha consimido la ruta por y la penalización de la ruta por atender clientes fuera de sus ventanas. Estas variables son actualizadas durante la exploración de las rutas y finalmente se penaliza el costo total en caso de haber atendido clientes fuera de tiempo.


\section{Estrategias}
El sistema cuenta de forma nativa con varias estrategias de exploración y selección, mencionadas en \ref{2-blueprint} que al ser combinadas generan muchos tipos de exploraciones distintas. Es posible también extender la generación de funciones de exploración al definir nuevos tipos de estrategias.

La definición de nuevas estrategias lleva dos pasos. Primero debe crearse la clase que representa la estrategia siendo definida. En dependencia del comportamiento esperado, la nueva clase puede heredar de clases auxiliares ya creadas o incluso de nuevas clases auxiliares que deben ser definidas. El segundo paso es implementar las especializaciones de métodos de cada tipo (inicializaciones dentro del \textit{let}, código dentro del ciclo, código de retorno, etc) necesarias que reciban como parámetros las nuevas clases creadas. Opcionalmente se puede asociar una instancia de la nueva clase a un parámetro global con el siguiente formato: \textbf{+\textit{name}-\textit{type}-strategy+}, donde \textbf{\textit{name}} es el nombre de la estrategia y \textbf{\textit{type}} es \textit{search} o \textit{selection}.


Por ejemplo, se creará una estrategia de selección nueva tal que se retorne un vecino aleatorio de una lista de vecinos encontrados, pero esa lista sólo tendrá soluciones cuya mejora superen cierto margen con respecto a la inicial. se define la clase \textbf{random-improvement-with-restricted-candidates-selection-strategy} y una instancia se asocia al parámetro global \textbf{+random-improvement-with-restricted-candidates-selection-strategy+}.

La nueva clase tiene un slot \textbf{aceptance} para el margen de aceptación cuyo valor estará entre 0 y 1. Se aceptarán soluciones que cumplan:

$$
cost_s < cost_{init-s} - cost_{init-s} * \textbf{aceptance} 
$$

Debe heredar de las siguientes clases auxiliares existentes:

\begin{itemize}
	\item \textbf{use-eval-graf}: Para generar soluciones con Árbol de vecindad y evaluarlas con grafo de evaluación.
	\item \textbf{return-best-solution}: Para crear y retornar una variable \textbf{best-solution}
\end{itemize}

Además, se debe crear la nueva clase auxiliar:

\begin{itemize}
	\item \textbf{has-restricted-candidates-for-best-neighbor}: Tiene un comportamiento parecido a \textbf{has-restricted-candidates-best-neighbor} pero los candidatos sólo serán agregados si superan cumplen la restricción de aceptación.
\end{itemize}

Luego deben implementarse las especializaciones de los métodos \textbf{generate-code-inside-let}, \textbf{generate-code-inside-loop} y \textbf{generate-code-outside-loop} tal que reciban como parámetro de estrategia de selección la nueva clase \textbf{random-improvement-with-restricted-candidates-selection-strategy}.

En las inicialicaciones dentro del \textbf{let} se inicializa una variable con nombre \textbf{candidates-for-best-neighbor}.

Dentro del ciclo se verifica si el costo de la solución inicial cumple con la restricción establecida y en caso positivo, se agrega esta a la lista \textbf{candidates-for-best-neighbor}.

Finalmente, fuera del ciclo se escoge aleatoriamente un vecino de \textbf{candidates-for-best-neighbor}, se asocia este a la variable \textbf{best-neighbor} y se asocia su costo a \textbf{best-cost}. Las variables \textbf{best-neighbor} y \textbf{best-cost} son creadas dentro del let en la especialización de \textbf{return-best-solution}. En caso de \textbf{candidates-for-best-neighbor} estar vacía, entonces no se hace nada y tanto \textbf{best-neighbor} como \textbf{best-cost} permanecen iguales.





