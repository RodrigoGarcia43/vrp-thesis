\chapter{Propuesta de solución}\label{chapter:Solution}

En este capítulo se describe el sistema implementado y cómo fueron unidas todas las piezas aisladas.

En \ref{2-org} se explica las ventajas del formato .org utilizado para desarrollar los códigos y exportarlos a archivos .lisp. También se explica la estructura general del sistema.

El sistema está separado en secciones.

En \ref{2-core} se presenta la sección central que representa los cimientos sobre los que se construyó el sistema.

En \ref{2-neigh} se presenta el Árbol de Vecindad y el generador de soluciones.

En \ref{2-eval} se presenta el Grafo de Evaluación.

En \ref{2-blueprint} se presenta la función generadora de código fuente para explorar vecindades utilizando combinaciones de estrategias.

\section{Formato org}\label{2-org}
El sistema está desarrollado sobre una serie de archivos con formato org (Lotus Organizer File). Org es un modo Emacs para guardar notas, mantener listas TODO, y hacer planificación de proyectos con un rápido y efectivo sistema de texto plano. Es también un sistema de publicación y autoría, que soporta trabajar con código fuente para programación literal e investigación reproducible \cite{TODO}.

Org facilita la organización de los archivos facilitando la separación de los datos en regiones. Los títulos de cada región comienzan con una cierta cantidad de símbolos "$\ast$". Las regiones se ramifican en subregiones de acuerdo a la cantidad de asteriscos.

El texto plano en los archivos contiene información y explicaciones de los elementos implementados. Además, Org-mode proporciona un número de funcionalidades para trabajar con código fuente, incluyendo la edición, evaluación y exportación de bloques código.

La estructura de bloques de código es como sigue:\\
\#+NAME: $<$name$>$\\
\#+BEGIN\_SRC $<$language$>$ $<$switches$>$ $<$header arguments$>$\\
$<$body$>$\\
\#+END\_SRC

Donde $<$name$>$ es una cadena usada para nombrar el bloque de código $<$language$>$ que especifica el lenguaje del bloque de código (lisp en este caso) $<$switches$>$ puede usarse para controlar la exportación del bloque de código, $<$header arguments$>$ puede usarse para controlar muchos aspectos del comportamiento de bloques de código, y el $<$body$>$ contiene el código fuente actual.

El argumento de cabezera $:tangle$ $<archivo>$ permite exportar los bloques de código al archivo seleccionado. en este caso, los códigos se exportan a archivos con formato \textit{.lisp} que son creados en la carpeta \textit{src} ubicada en la raíz del sistema. Los archivos de \textit{src} contienen todas las funciones que deben ser importadas para utilizar el sistema. El proceso de inicialización del sistema se explica en \ref{chapter:Tutorial}.

La implementación sistema está dividida en cuatro secciones principales.

\begin{itemize}
	\item \textit{Core}: Contiene funciones para inicializar el sistema, datos de instancias conocidas de VRP y criterios clásicos, implementaciones de algoritmos de búsqueda local y definiciones de clases a partir de las cuales es posible definir un problema de VRP en lenguaje Common Lisp. También se definen las operaciones a partir de las que se forman criterios de vecindad.
	\item \textit{Neigh}: En los archivos de \textit{neigh} se implementa el árbol de vecindad a partir del cual se generan soluciones.
	\item \textit{Eval}: En \textit{eval} se implementa el grafo de evaluación con el que se evalúan soluciones.
	\item \textit{Blueprint}: Esta sección contiene código para generar funciones de exploración a partir de combinaciones de estrategias de exploración y selección.
\end{itemize}

Para resolver computacionalmente un problema, el primer paso es definir una forma para describirlo a partir de código. La sección \textbf{core} es la base del sistema. 


\section{Vrp-Core}\label{2-core}

En \textit{core} se definen funciones útiles para inicializar fácilmente el sistema, datos de instancias conocidas de VRP y criterios clásicos útiles para la investigación, así como la implementaciones de algoritmos de búsqueda local que utilizan funciones del sistema para la exploración de vecindades. Estos elementos serán analizados a profundidad en \ref{chapter:Tutorial}.

Se definen además una serie de clases a partir de las cuales es posible definir problemas de VRP con sus especificaciones y restricciones en lenguaje Common Lisp. También se crean clases para representar operaciones que conforman los criterios de vecindad (como seleccionar ruta o insertar cliente).

Los elementos que forman un problema están definidos en clases. Los elementos que presenta un VRP en sus versiones más simples son:

\begin{enumerate}
	\item Clientes
	\item Vehículos
	\item Depósitos
\end{enumerate}

Siendo las variaciones de VRP sólo limitadas por la imaginación, las características específicas de estas variaciones son también virtualmente infinitas. Por ejemplo, para el Problema de Enrutamiento de Vehículos con Múltiples productos \cite{TODO}, se ha definido también \textit{Producto como una característica}. Cómo ampliar el sistema para definir otras variantes de VRP se discute en \ref{chapter:Tutorial}. También se tienen clases para representar rutas y soluciones.

Para crear una instancia de un problema de VRP deben instanciarse elementos acordes a sus especificidades. Por ejemplo, una instancia de VRP clásico utiliza clientes básicos, mientras que una de CVRP trabaja con clientes con demanda.

Se utiliza la herencia múltiple de Common Lisp para definir las características de cada elemento mediante clases abstractas. Por ejemplo, un vehículo de CVRP (\textbf{cvrp-vehicle}) se define como una clase que hereda de las clases abstractas: \textbf{basic-vehicle} (vehículo básico), \textbf{capacity-vehicle} (vehículo con capacidad) y \textbf{cargo-vehícle} (vehículo que lleva carga).

Una vez definidas las clases que representan un problema y sus elementos, estas pueden ser utilizadas en el resto de las secciones del sistema. En la sección d \textit{neigh} se define el Árbol de Vecindad con generador de soluciones.

\section{Vrp-neigh}\label{2-neigh}
En esta sección se define el Árbol de vecindad como una clase que almacena sus propiedades, tales como el problema, la solución y el criterio de vecindad que este representa. Se define también una clase para las soluciones de conteo (explicada en \ref{2-Hector}) distinta a la clase solución de \textit{vrp-core}. 

La función \textbf{build-neighborhood-tree} recibe una solución y una lista de operaciones que representa el criterio de vecindad que se quiere explorar. Esta función instancia, construye y retorna un Árbol de vecindad

Dado un Árbol de Vecindad $T$ con cardinalidad $k$, a cada elemento perteneciente a la vecindad representada se le hace corresponder un número entero \textit{index}, con:
$$
1 <= index <= k
$$

Además, la función \textbf{from-index-to-coordinate-list} permite obtener el vecino correspondiente a cada índice. Los vecinos son retornados como una lista que contiene las operaciones a realizar sobre la solución inicial. La forma de generación de soluciones durante una exploración varía de acuerdo a la estrategia de exploración seleccionada. Por ejemplo, se puede generar todos los vecinos de forma exhaustiva o, en cambio generar una cantidad fija $c$ de vecinos aleatorios. 

Las funciones de exploración (como \textbf{exhaustive-exploration} o \textbf{random-exploration}) devuelven, a partir de un árbol de vecindad una función $lambda$ generadora de soluciones. Cada vez que se invoque esta función, se recibirá la lista de operaciones que representan una solución, o \textbf{nil} en caso de haber llegado a la condición para terminar la exploración. Esta condición varía según el tipo de exploración, por ejemplo, para la exploración exhaustiva es generar $k$ soluciones mientras que en la exploración aleatoria es generar $c$. 

A partir del árbol de operaciones se hizo generadores de soluciones que construyen soluciones vecinas de una operación inicial en forma de lista de operaciones. Al aplicar las operaciones de una lista sobre la solución inicial se puede formar una nueva solución que, al ser evaluada retorna un costo. El costo de la nueva solución se utiliza durante la exploración para buscar optimizaciones. Evaluar una solución es un proceso potencialmente costoso y hacerlo para soluciones de vecindades con cardinalidades grandes puede ser computacionalmente intractable. Para evaluar eficientemente las soluciones generadas se utiliza un Grafo de Evaluación.

\section{Vrp-eval}\label{2-eval}
En esta sección se implementa el Grafo de Evaluación. La clase \textbf{eval-graph} almacena el estado del grafo y lleva constancia de sus nodos. También, mediante tablas de hash hacen corresponder cada nodo con su respectivo elemento del problema. Se define también una clase para cada tipo de nodo existente, de alto y bajo nivel.

Para inicializar un grafo se necesita sólo de una solución inicial. La inicialización consiste en convertir los elementos de una solución (clientes, depósitos, vehículos) en nodos de alto nivel que están inicialmente aislados (un grafo de evaluación es bipartito \cite{Hector}, sólo se conectan nodos de diferentes tipos). Los métodos \textbf{convert-to-node} utilizan el sistema CLOS para crear nodos de alto nivel correspondientes a los elementos pasados como parámetros. Por ejemplo, al llamar a conver-to-node con un objeto de tipo \textbf{basic-client} como argumento, se crea un nuevo nodo que se agrega al grafo y, en una tabla de hash, se hace corresponder al objeto con el nuevo nodo.

Los nodos creados durante la inicialización del grafo se clasifican en nodos de tipo \textit{entrada}. Terminar la construcción del grafo requiere de la ejecución del código de evaluación de la solución inicial. A partir de este código se crean los nodos de alto nivel restantes (nodos de tipo \textit{salida}). También se crean y evalúan los nodos de bajo nivel que unen a los nodos de alto nivel. Nótese que los nodos de tipo \textit{salida} pueden constituir salidas parciales que son al mismo tiempo \textit{entradas} de otros nodos de bajo nivel.


El código de evaluación se hace utilizando los métodos de construcción de grafo definidos en el sistema tales como:

\begin{itemize}
	\item \textbf{defvar}: Recibe un nombre y un valor inicial. Inicializa un nuevo nodo de alto nivel de tipo \textit{salida} y lo asocia (mediante una tabla de hash) al nombre recibido.	
	\item \textbf{increment-distance}: Recibe dos clientes, una variable asociada a un nodo tipo \textit{salida} y la matriz de distancia. Crea y evalúa un nodo de bajo nivel que recibe los nodos asociados a los clientes como entrada e incrementa el valor almacenado en el nodo salida un una cantidad igual a la distancia entre ellos.
	\item \textbf{decrement-demand}: Recibe un cliente y una variable asociada a un nodo tipo \textit{salida}. Crea y evalúa un nuevo nodo de bajo nivel que recibe como entrada el cliente y decrementa el valor del nodo \textit{salida} en una cantidad igual su demanda
	\item \textbf{increment-value}: recibe una variable y un valor. Incrementa el valor almacenado en el nodo asociado al nombre en una cantidad igual al valor recibido.
	\item \textbf{apply-penalty}: Recibe dos variables asociadas a nodos \textit{salida} y un factor de penalización. Crea y evalúa un nuevo nodo de bajo nivel que recibe como entrada el nodo asociado a la primera variable y, en caso de este almacenar un valor negativo, aumenta el valor almacenado en el nodo asociado a la segunda variable en una cantidad dependiente del factor de penalización.
	\item \textbf{return-value}: Recibe el nombre de una variable y marca su nodo asociado en el grafo como el nodo contenedor del valor de salida. A partir del punto en que se invoca, el slot \textit{output} del grafo referenciará a este nodo.
\end{itemize}

Todas las funciones de construcción de grafo reciben también como parámetro la instancia del grafo sobre el que se está trabajando.

La ampliación del sistema con nuevos métodos se discutirá en \ref{chapter:Tutorial}. A continuación se muestra el código de evaluación de una solución de VRP clásico.


\begin{lstlisting}
(progn
	(def-var total-distance 0 graph)
	(loop for r in (routes s1) do 
		(progn
			(def-var route-distance 0 graph)
			(loop for c in (clients r) do 
				(progn
					(increment-distance (previous-client r) c route-distance dist-mat graph)
					(setf (previous-client r) c)))
			(increment-value total-distance route-distance graph)
		(return-value total-distance graph)))

\end{lstlisting}

En la línea \textit{1} se usa def-var para inicializar la variable que almacenará el costo total de la solución. Luego se analizan las rutas de la solución. Por cada ruta se define una variable nueva que almacena su costo (línea \textit{5}). Entonces se analizan los clientes de la ruta actual. Se incrementa el costo de la ruta en una cantidad igual a la distancia del cliente actual y el cliente previo (el cliente previo inicial de la ruta es el depósito)(líneas \textit{9}). Después de analizada cada ruta se aumenta el costo total de la solución en una cantidad igual a los costos de las mismas (líneas \textit{12}). Finalmente se retorna la distancia total.

Una vez construido el grafo a partir de la solución inicial y el código de evaluación, es posible evaluar eficientemente soluciones vecinas ejecutando la función \textbf{do-core-operations}. Esta función recibe una lista de operaciones y el grafo sobre el cual estas serán realizadas. Las operaciones se realizan en orden, los nodos de alto nivel se insertan o retiran y los nodos de bajo nivel correspondientes son evaluados o desevaluados. Por ejemplo, dado el grafo representado por \ref{fig:eval-graph-2}, al ejecutar el siguiente código:

\begin{lstlisting}
(do-core-operations graph (list (op-select-client 1 1 0)
													(op-insert-client 2 3 0))
\end{lstlisting}

Se obtiene el grafo representado en \ref{fig:eval-graph-4}. En este punto, todos los nodos necesarios fueron evaluados y desevaluados, y, por tanto, en el slot \textit{output} del grafo (Que en este caso hace referencia al nodo \textit{cost}) está almacenado el costo total de la nueva solución.

Para deshacer operaciones realizadas sobre un grafo se ejecuta el método \textbf{undo-core-operations}. Por ejemplo, al ejecutar el siguiente código sobre el grafo representado en \ref{fig:eval-graph-4}:

\begin{lstlisting}
(un-core-operations graph (list (op-select-client 1 1 0)
												(op-insert-client 2 3 0))
\end{lstlisting}

Se obtiene nuevamente el grafo representado por \ref{fig:eval-graph-2} y el valor almacenado en \textit{cost} vuelve a ser el costo total de la solución inicial.

Una vez definidas las funciones que generan y evalúan soluciones, sólo resta utilizarlas para la exploración de vecindades.

\section{Vrp-blueprint}\label{2-blueprint}
Combinar diferentes estrategias de exploración y selección puede dar como resultado numerosas exploraciones con resultados distintos. Programar cada una de estas exploraciones es un proceso con alto consumo de tiempo. En \cite{Heidy} se propone una forma de automatizar el proceso de construcción de funciones exploradoras. Se aprovecha las ventajas del lenguaje Common lisp para generar código fuente y  el sistema CLOS, y se genera el código de una función lambda. El código generado depende de los tipos de estrategias seleccionados.

La función \textbf{make-neighborhood-criterion} (función generadora) recibe un criterio de vecindad (en forma de lista de operaciones), una estrategia de exploración y una estrategia de selección. Se retorna una función lambda que al ser invocada retorna la solución encontrada durante la exploración en forma de objeto de tipo \textbf{solution} o \textit{nil} en caso de no encontrarse ninguna.

La función lambda (función de exploración) recibe como entrada la solución inicial, el problema y el Grafo de Evaluación que representa a la solución inicial. Luego de invocada, la instancia del grafo tendrá aplicadas las operaciones que dan como resultado la solución encontrada, que pasará a ser la solución inicial de próximas operaciones realizadas sobre el grafo. En caso de no encontrarse solución, el grafo no cambiaría.

Las estrategias se definen como clases cuyas instancias son pasadas como argumentos a la función generadora. Se utilizan métodos que reciben estas instancias para generar código. El código que se genera depende de los métodos especializaciones que se ejecutan y estos métodos dependen de los tipos de los argumentos recibidos, como se explicó en \ref{2-Lisp}.

A continuación se presentan algunas clases que representan estrategias de exploración y selección:\\

\textit{Exploración:}
\begin{itemize}
	\item \textbf{exhaustive-neighborhood-search-strategy}: Exploración exhaustiva.
	\item \textbf{random-neighborhood-search-strategy}: Exploración aleatoria.
\end{itemize}

\textit{Selección:}
\begin{itemize}
	\item \textbf{best-improvement-search-strategy}: Selección de mejor solución.
	\item \textbf{first-improvement-search-strategy}: Selección de primera mejora.
	\item \textbf{random-improvement-with-candidates-selection-strategy}: Selección aleatoria de una mejora entre los elementos de una lista con las mejoras encontradas.
	\item \textbf{random-improvement-selection-strategy}: Selección aleatoria de una mejora en base a una probabilidad.
\end{itemize}

La extensión del sistema a partir de nuevas estrategias se discutirá en \ref{chapter:Tutorial}.

Durante la definición de cada clase se crea un parámetro global que como valor tiene una instancia de su respectiva clase. Por ejemlo, el parámetro +exhaustive-search-strategy+ tiene como valor asociado una instancia de \textbf{exhaustive-neighborhood-search-strategy}.

El siguiente código de ejemplo genera una función lambda que explora el criterio \textit{rarb} de forma exhaustiva y retornando mejor solución:


\begin{lstlisting}
		 (setf rarb (make-neighborhood-criterion 
								`((select-route r1)
									(select-client c1 from r1)
									(select-route r2)
									(insert-client c1 in r2))
								+exhaustive-search-strategy+ 
								+best-improvement+)))
\end{lstlisting}

Luego, la función generada \textbf{rarb} puede ser invocada para ejecutar la exploración y obtener el mejor vecino encontrado.

\begin{lstlisting}
	 (setf best-neighbor (funcall rarb solution	problem	graph))
\end{lstlisting}	












